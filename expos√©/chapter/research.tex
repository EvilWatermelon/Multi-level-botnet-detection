\section{Stand der Forschung}
\label{sec:research}

\subsection*{Methoden zur Erkennung von Botnetzen}
% - SDN, Edge Computing etc. Referenz muss hierhin und einzelne Techniken noch erklärt werden
Xing et al. \cite{Xing2021SurveyOB} stellen eine zusammengefasste Unterteilung von Methoden zur Botnetzerkennung vor, wobei sie nicht konkreter auf die Methoden eingehen. Speziell auf Botnetze im IoT
Gebiet fassen Wazzan et al. \cite{Wazzan2021InternetOT} Methoden anhand von Studien zusammen, die zur Erkennung eingesetzt werden. Für den aktuellen Stand der Forschung nutzt dieses Exposé deshalb die
ausgearbeiteten Zusammenfassungen von \cite{Wazzan2021InternetOT}, um einen Ansatz zu finden, wie die Botnetzerkennung stattfinden könnte. Im Zusammenhang mit der Botnetzerkennung findet ein Großteil
der Forschung zu diesem Gebiet im Zusammenhang mit maschinellem Lernen (ML) statt, wie unter anderem zum Beispiel \cite{SAlrayes2022ModelingOB,DBLP:journals/comcom/Alani22,Habtamu2022ASR, D2022ARO}
zeigen. Wazzan et al. schlagen die Kombination mehrerer Erkennungsmethoden vor, womit sich die Dissertation auch beschäftigen soll. Nach den genannten Studien schlagen Wazzan et al. folgende
Erkennungsmethoden vor: Software Defined Networks (SDN), Edge Computing, Blockchain, Fog Computing und Network Function Virtualization (NFV). Im Folgenden wird der aktuelle Stand der Forschung zu
jeder dieser Technologien betrachtet. \\ \\ Zha et al. \cite{DBLP:conf/cns/Zha0GMC19} beschreiben eine Bibliothek, welche eine Botnetzerkennung anhand eines SDNs durchführt.
Bei dem Konzept des Frameworks handelt es sich darum, das Botnetz und C\&C-Kanal Aktivitäten zu identifizieren, damit die Robustheit der Erkennung erhöht wird. Zum Aufbau gehört ein SDN
Controller, welcher ein Modell des ML enthält, dass über Erkennung und Überwachung trainiert wird. Die Überwachung des Netzwerks findet über Software Switches statt, welche vollen Zugang zum
Netzwerkverkehr von virtuellen Maschinen (VM) haben. Das Modell wird für das komplette Netzwerk verwendet, um Bot Aktivitäten zu erkennen, während für einzelne Server, dessen VMs mit
einem Software Switch kommunizieren, lokale Netzwerküberwachung stattfindet. Über die einzelnen Software Switches, findet die C\&C Identifizierung statt. Dabei sind speziell Peer-to-Peer
Verkehr, HTTP und IRC Protokolle im Fokus. Das Modell ist ein neuronales Netzwerk (NN), welches kompromittierte Hosts findet anhand von aufgebauten Verbindungen des Hosts. Die Entscheidung,
ob ein Host zu einem Bot wird, entscheidet das NN über einen festgelegten Schwellenwert. Mit einer einzelnen positiven Identifizierung ist aber dennoch nicht eindeutig geklärt, dass der
Host auch wirklich ein Bot ist, da er zum Beispiel nicht unbedingt mit dem C\&C-Server kommuniziert. An dieser Stelle wird der Host dennoch dauerhaft weiter beobachtet. Der aktuelle Stand der
Forschung zeigt hier, dass die Erkennung nicht während der Scan-Phase durchführt wird, womit sich ein Ziel ergibt Botnetzerkennung in frühen Botnetzphase durchzuführen, was in der Dissertation
verfolgt werden soll und weiter in Kapitel \ref{sec:goals} erläutert wird. Zusätzlich zu dem beschriebenen Framework erläutern Negera et al. \cite{DBLP:journals/sensors/NegeraSDMA22} verschiedene
Arbeiten zur Erkennung von Botnetzen in SDNs über Modelle des ML. \\ \\ Neben der Erkennung mit SDNs findet in der aktuellen Forschung auch die Erkennung von Botnetzen über Edge Computing
Architekturen statt, wie \cite{gromov2022edge} in ihrer Arbeit zeigen. Bei dieser Erkennung verwenden Gromov et al. in einer Edge Computing Architektur ein Convolutional Neural Network (CNN). Der
Netzwerkverkehr der einzelnen IoT-Geräte wird an ein NVIDIA Jetson Nano Entwickler Kit zur Botnetzerkennung geleitet. Dabei werden anhand von Computer Vision, Bilder zur Erkennung erzeugt, welche in
auffällige und unauffällige Bilder klassifiziert werden, um die Performance zu erhöhen sowie das Netzwerk zu verkleinern. \\ \\ Bei der blockchainbasierten Erkennung geht es darum, eine Blockchain zu
implementieren und in diesem Netzwerk ein IoT-Netzwerk auf Botnetze zu überwachen. Dabei werden unterschiedliche Methoden innerhalb der Blockchain implementiert und mit den Vorteilen der Blockchain
verwendet. Sagirlar et al. \cite{DBLP:journals/corr/abs-1809-10775} nutzen diese Methode, indem sie innerhalb der Blockchain eine Peer-to-Peer Botnetzerkennung implementieren. Salim et al.
\cite{DBLP:journals/sensors/SalimATPP22} erläutern ein Framework, welches Digital Twins und Packet Auditor in einer Blockchain zur Erkennung einsetzen. \\ \\ Neben Blockchains wird auch Fog 
Computing zur Botnetzerkennung eingesetzt. \cite{Lawal2020AnAM} stellen ein Framework vor, welches Anomalien in einem IoT Netzwerk erkennt, aufgebaut als Fog Computing Architektur, anhand von
signatur- und anomaliebasierten Erkennungsmethoden. Die signaturbasierte Methode greift auf eine Datenbank zu, welche IP-Adressen enthält, die als Angriffsursprung identifiziert sind. Die
anomaliebasierte Methode klassifiziert anhand eines Extreme Gradient Booster (XGBoost) Algorithmus \cite{DBLP:journals/corr/ChenG16} zwischen auffälligem und unauffälligem Netzwerkverkehr. Der
Netzwerkverkehr fließt zu Beginn durch ein signaturbasiertes IDS, um die IP-Adresse mit der Datenbank zu vergleichen. Sollte nichts gefunden werden, fließt der Verkehr durch ein anomaliebasiertes IDS
und klassifiziert ihn entsprechend. \\ \\ Das letzte Framework erläutern Kim et al. \cite{DBLP:journals/cn/KimNPSS19} in ihrer Arbeit zu einer SDN Umgebung mit integrierten Netzwerk-Funktionen auf
NFV, welche in den IoT-Geräten integriert sind. Das Konzept aus der Arbeit von Kim et al. beschreibt ein IoT-Netzwerk, welches für jede einzelne Komponente eigene Sicherheitsrichtlinien vorgibt. Die
Architektur des Netzwerks beginnt mit einer \textit{Control Plane}, welches für das Management des Netzwerks verantwortlich ist. Die \textit{Control Plane} besteht aus fünf Komponenten:
\textit{Datapath}, welches für die Kommunikation zwischen \textit{Control Plane}, \textit{Function Plane} und \textit{Data Plane} (Verbindung zu den IoT-Geräten) verantwortlich ist. Die zweite
Komponente ist der \textit{Protocol Parser} der Nachrichten von der \textit{Function-} und \textit{Data Plane} empfängt und zum \textit{Core Module} weiterleitet. Das \textit{Core Module} verwaltet
Netzwerkkomponenten, stellt Sicherheitsfunktionen zur Verfügung und erkennt Anomalien bei den Sicherheitsrichtlinien. Der \textit{Event Manager} stellt Event basierte Kommunikation zur
Verfügung. Die letzte Komponente sind \textit{Applications} welche über ein \textit{programmable interface} implementiert sind und für die Verarbeitung von den IoT-Geräten zuständig sind. Die
\textit{Function Plane} ist für die Sicherheit der IoT-Geräte zuständig, da diese durch begrenzte Leistung anspruchsvolle Sicherheitsfunktionen nicht verarbeiten können. Diese Sicherheitsfunktionen
werden über NFV Techniken ausgeführt, damit die Funktionen virtualisiert zur Verfügung stehen. Die Kommunikation findet über das von Kim et al. vorgestellte SODA Protokoll statt, damit zum Beispiel
Richtlinien richtig verwaltet werden können. Wenn zum Beispiel neue Geräte dem Netzwerk hinzugefügt werden, dann werden anhand von eventbasierten Prozessen diese auf zum Beispiel Sicherheitsbedenken
geprüft. Sollte über das Netzwerk ein Angriff stattfinden, so wird das an den Richtlinien erkannt und diese dann angepasst. \\ Bei der Implementierung wurde die \textit{Control-} und \textit{Function
Plane} als SDN Controller implementiert, das \textit{Data Plane} als Kommunikation zwischen dem Controller und den IoT-Geräten. Zur vereinfachten Ressourcenverwaltung nutzen Kim et al. NFV bei jedem
IoT-Gerät. \\ \\ Aus den Arbeiten der aktuellen Forschung geht hervor, dass bei der Verwendung von ML die eingesetzten Algorithmen immer zu Ergebnissen führen, bei denen das Modell eine Genauigkeit von über 90\% erreicht. Dabei gehen die Arbeiten nicht weiter darauf , warum der entsprechende Algorithmus verwendet wird und welche Vorteile dieser bringt. Zudem erkennen die Arbeiten die Botnetze
immer zu den späteren Phasen eines Botnetzes, womit die Arbeiten keine Prävention von Botnetzen durchführen. Des weiteren nutzen die Botnetzerkennungen nur einzelne Methoden und kombinieren diese nicht miteinander, um zum Beispiel die Nachteile einer Methode mit einer weiteren auszugleichen. Wenn die Netzwerkarchitekturen wie zum Beispiel SDNs mit in die Erkennung von IoT Botnetzen mit einbezogen werden, zeigt die aktuelle Forschung, dass dies zu einer besseren Leistung der Botnetzerkennung führt, um so zum Beispiel Latenzen zu verringern. Aus Kapitel \ref{sec:theory} sowie dem aktuellen Stand der Forschung geht hervor, dass \cite{Xing2021SurveyOB} und \cite{Wazzan2021InternetOT} als mögliche Ansätze für die Botnetzerkennung verwendet werden können, da diese Arbeiten konkretere Methoden und Vorschläge machen, um die Botnetzerkennung zu optimieren.

\subsection*{Multi-Level Botnetzerkennung}
Neben der Forschung zur Botnetzerkennung anhand einzelner Methoden soll dieses Kapitel den aktuellen Stand der Forschung zur Botnetzerkennung anhand kombinierter Methoden vorstellen. Rahal et al. 
\cite{DBLP:journals/jnsm/RahalKGCG22} beschreiben ein Framework, welches auf Netzwerkebene und Aktiviäten in Fahrzeugen beobachtet. Eine weitere Studie untersucht die Kombination eines Algorithmus,
um Merkmale in einem Suchraum zu finden, mit einem Modell des ML zur Klassifizierung von Bedrohungen in einem Flying ad hoc network \cite{DBLP:journals/bdcc/AbdulsattarAGKA22}. Almutairi et al.
\cite{DBLP:journals/jcnc/AlmutairiMAA20} implementieren eine Botnetzerkennung auf Basis einer Hostbasierten und netzwerkbasierten Erkennung. Der Fokus bei der Erkennung liegt dabei auf den frühen
Phasen eines Botnetzes, was genauer bedeutet, die Bots bei der Verbreitung zu erkennen. Die hybride Botnetzerkennung wird dazu genutzt, um \textit{abnormales Verhalten} zu erkennen und falls über 
die Netzwerkanalyse nichts erkannt wird, kann die hostbasierte Analyse zusätzlich über weitere Regeln unterstützen. Um Schwachstellen zu sammeln, setzen die Autoren einen Honeypot ein, welcher über
das Internet Angreifer dazu bringen soll, das System anzugreifen. Ein Controller überwacht das Netzwerk und steuert VMs. Zudem ist eine Datenbank für Malware Signaturen eingesetzt. Um das Verhalten
von Bots zu beobachten, wird in einer VM ein Analysewerkzeug genutzt. Eine weitere VM führt eine dynamische Analyse durch. In der ersten Phase sammelt das System Malware Signaturen, welche in einer
zweiten Phase klassifiziert werden an von ML. Almutairi et al. erläutern, dass die Netzwerkanalyse wichtig ist, da die Bots über das Internet kommunizieren, um zum Beispiel Nachrichten an den C\&C-
Server zu schicken. \\ \\ Auch bei der Multi-Level (auch hybriden) Botnetzerkennung liegt der Fokus auf einer Phase und auch überwiegend auf die späteren Phasen. Zusätzlich legen die Arbeiten auch in
den meisten Fällen separat den Fokus auf C\&C-Server Erkennung oder der Erkennung von Bots.


% - https://pdfs.semanticscholar.org/3b6c/9211ac49aee6bc2604b3c414d840e43ac58f.pdf
% - https://e-space.mmu.ac.uk/627198/1/09241019.pdf
% - https://pdfs.semanticscholar.org/4b14/c20e652c221bc415431b529289f9ce3637d7.pdf
