\section{Stand der Forschung}
\label{sec:research}

Im Zusammenhang mit der Botnetzerkennung findet ein Großteil der Forschung zu diesem Gebiet im Zusammenhang mit Maschinellem Lernen (ML) statt wie
unter anderem zum Beispiel \cite{SAlrayes2022ModelingOB,DBLP:journals/comcom/Alani22,Habtamu2022ASR} zeigen.
Wazzan et al. \cite{Wazzan2021InternetOT} schlägt die Kombination mehrerer Technologien vor womit sich die Dissertation auch beschäftigen soll.
Daher sollen neben der Erkennung über ML auch weitere Technologien mitbetrachtet werden. Wazzan et al. fasst anhand verschiedener Studien folgende 
weitere Technologien zusammen: Software Defined Network (SDN), Edge Computing, Blockchain, Fog Computing und Network Function Virtualization (NFV). 
Im folgenden wird der aktuelle Stand der Forschung zu jeder dieser Technologien betrachtet. \\ \\ Zha et al. \cite{DBLP:conf/cns/Zha0GMC19} beschreiben 
eine Bibliothek, welche Botnetz Erkennung anhand eines SDNs durchführt. Zusätzlich zu dem beschriebenen Framework, erläutern Chen et al. \cite{Chen2017BotGuardLR}
ein weiteres Framework unter der Verwendung von SDNs.
