\section{Forschungsfragen und Aufbau des Exposés}

Botnetze im Internet of Things (IoT) Gebiet haben in der modernen Welt sehr stark an Bedeutung gewonnen. Gerade im privaten 
Haushalt finden sich immer mehr IoT-Geräte die durch Botnetze wie Mirai und Hajime infiziert werden können. Daher ist es 
sehr wichtig, dass diese Botnetze durch eine Erkennung identifiziert und abgewehrt werden. Die Dissertation soll sich mit 
der Erkennung von Botnetzen beschäftigen und die Identifikation optimieren.

\subsection*{Forschungsfragen für die Dissertation}

\newlist{questions}{enumerate}{2}
\setlist[questions,1]{label=\textsf{\textbf{Q\arabic*:}},ref=RQ\arabic*}
\setlist[questions,2]{label=(\alph*),ref=\thequestionsi(\alph*)}

Die Dissertation soll spezifisch die folgenden Forschungsfragen beantworten:  

\begin{questions}
    \item Wie lassen sich Botnetze während der Verbreitungsphase erkennen? \label{itm:q1}
    \item Wie lassen sich Botnetze während der Scan-Phase erkennen? \label{itm:q2}
    \item Welche Kombinationen von bereits bekannten Methoden führen zu einer Verbesserung der Erkennung von Botnetzen? \label{itm:q3}
    \item Welche Kombinationen von bereits bekannten Techniken führen zu einer Verbesserung der Erkennung von Botnetzen? \label{itm:q4}
    \item Wie lässt sich die Erkennung von Botnetzen optimieren, wenn mehrere Erkennungsstufen verwendet werden? \label{itm:q5}
    \item Wie sieht ein Datensatz aus, der zur Multi-Level Erkennung verwendet wird? \label{itm:q6}
\end{questions}

In Frage \ref{itm:q1} und \ref{itm:q2} soll es um die Erkennung von Botnetzen während der ersten beiden Phasen gehen. Was die Phasen sind erklärt das nachfolgende Kapitel. 
\\ \\
In den folgenden Kapitel \ref{sec:theory} werden die theoretischen Hintergründe zum Thema der Dissertation und verschiedene Prozesse zur Erkennung von Botnetzen, sowie 
mehrere Botnetze die in Fallstudien eingesetzt werden sollen beschrieben. Kapitel \ref{sec:research} erläutert den aktuellen Stand der Forschung zur Erkennung von Botnetzen,
sowie zu den Themen die in der Dissertation behandelt werden sollen. In Kapitel \ref{sec:goals} werden die Ziele der Dissertation aufgestellt. Das Kapitel \ref{sec:methods}
erklärt wie in der Dissertation vorgegangen werden soll, um ein Konzept mit einer dazugehörigen Implementierung zu erstellen sowie den geplanten Aufbau eines Laborexperiments
und mögliche Fallstudien. Das letzte Kapitel \ref{sec:plan} stellt einen Zeitplan vor sowie die Struktur, die die Dissertation haben soll.