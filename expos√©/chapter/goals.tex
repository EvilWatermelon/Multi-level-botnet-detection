\section{Ziel der Dissertation}
\label{sec:goals}

Mit der Dissertation soll die Botnetzerkennung optimiert werden. Ziel ist es, ein Konzept zu erstellen, zusammen mit einer Implementierung.
Das erste Ziel soll ein Konzept zur Botnetzerkennung werden, welches zu jeder Phase eines Botnetzes, mehrere Erkennungslevel integriert, die aus einer Kombinationen 
mehrerer Erkennungsmethoden besteht. Der aktuelle Stand aus der Forschung zu den einzelnen Erkennungsmethoden zeigt außerdem, dass die aktuellen Methoden sich eher 
auf die späteren Phasen eines Botnetzes fokussieren. Daher ist ein weiteres Ziel der Fokus auf die Entwicklung zur Erkennung der früheren Phasen der Botnetze.
Das Ziel der Erkennungsmethoden ist es die Methoden herauszuarbeiten, welche die wenigsten false positives erzielen um diese dann miteinander zu kombinieren. Welche 
Methoden genau eingesetz werden, soll mit dieser Dissertation zu Beginn herausgefunden werden. Zusätzlich sollen verschiedene Kombinationen aus Techniken eingesetzt werden
, wie z.B. Flow- und Hostbasierte Techniken. Die Techniken sollen die Multi-Level Erkennung darstellen und zu jeder Phase separat ausgearbeitet werden, in deren Kombinationsmöglichkeiten.
Ein weiteres Ziel ist, ob die Erkennung auch unbekannte Botnetze identifizieren kann indem Fallstudien durchgeführt werden. \\ Ziel der Dissertation ist es nicht, aus der 
Erkennung Maßnahmen zur Verteidigung gegen Botnetze zu treffen. Die Dissertation soll sich ausschließlich auf die Erkennung konzentrieren. 