\section{Forschungsdesign und Methodik}
\label{sec:methods}

Zu Beginn des Konzepts um die Erkennung von Botnetzen zu optimieren, soll eine Ausarbeitung und Zusammenfassung von Methoden 
durchgeführt werden. Anhand der Zusammenfassung soll erkennbar werden, welche Methoden die besten Ergebnisse liefern 
und welche Kombination von Methoden und Techniken am plausibelsten sind. Die Dissertation geht anschließend mit einem 
Laborexperiment einher. Um die Erkennung eines Botnetzes zu testen sollen verschieden typische IoT-Geräte aufgebaut werden, 
die von Botnetzen wie Mirai angegriffen werden. Dazu ist es nötig, dass entsprechende Geräte ausgewählt werden, die von den 
Botnetzen infiziert werden können. \\ \\ In einer Fallstudie ist vorgesehen, Botnetze auszuführen um entsprechend Daten
zu sammeln, damit eine aussagekräftige Auswertung zu plausiblen Ergebnissen führt. \\ \\ Das Laborexperiment soll ein lokales 
Netzwerk mit gängigen Smart Home Geräten darstellen, die in einem privaten Haushalt verwendet werden. Wie schon in \ref{sec:theory}
erläutert, soll Mirai und Hajime entsprechend eingesetzt werden. Im optimalen Fall, soll für die Dissertation auch ein eigenes Botnetz
implementiert werden, um zu prüfen, ob die Botnetz Erkennung auch unbekannte Botnetze erkennt.
