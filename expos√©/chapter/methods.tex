\section{Forschungsdesign und Methodik}
\label{sec:methods}

Zu Beginn soll eine Ausarbeitung und Zusammenfassung von Methoden zur Botnetzerkennung durchgeführt werden. Anhand der Zusammenfassung soll 
erkennbar werden, welche Methoden die bestmöglichen Ergebnisse liefern und welche Kombinationen von Methoden am plausibelsten sind.
Daraus soll ein Konzept gebildet werden, welches Hybride Erkennungsmethoden für die frühen Phasen eines Botnetzes in der Theorie beschreibt.
Anschließend soll mit einem Laborexperiment das Konzept implementiert werden. Das Laborexperiment soll ein lokales Netzwerk mit gängigen Smart Home 
Geräten darstellen, die in einem privaten Haushalt verwendet werden. Um die Erkennung eines Botnetzes zu testen sollen verschiedene typische 
IoT-Geräte aufgebaut werden, die von Botnetzen wie zum Beispiel Mirai angegriffen werden. Dazu ist es nötig, dass entsprechende Geräte ausgewählt werden, 
die von bekannten Botnetzen infiziert werden können. \\ \\ In einer Fallstudie ist dann vorgesehen, Botnetze auszuführen, um entsprechende Daten
zu sammeln, damit eine aussagekräftige Auswertung zu plausiblen Ergebnissen führt. Um die Ergebnisse aus der Fallstudie überprüfen
zu können, sollen diese mit Ergebnissen aus anderen vergleichbaren Arbeiten in Relation gesetzt werden. \\ Wie schon in \ref{sec:theory}
erläutert, sollen unter anderem Mirai und Hajime eingesetzt werden. Im optimalen Fall, soll für die Dissertation auch ein eigenes Botnetz
implementiert werden, um zu prüfen, ob die Botnetz Erkennung auch unbekannte Botnetze erkennt.
