\section{Forschungsfragen und Aufbau des Exposés}

Botnetze im Internet of Things (IoT) haben in der modernen Welt sehr stark an Bedeutung gewonnen. Gerade IoT-Geräte, die
durch Botnetze wie Mirai und Hajime infiziert werden können, sind davon stark betroffen \cite{DBLP:conf/uss/AntonakakisABBB17}.
Daher ist es sehr wichtig, dass diese Botnetze durch eine Erkennung identifiziert und abgewehrt werden. Die Dissertation soll 
sich mit der Erkennung von Botnetzen beschäftigen und die Identifikation optimieren.

\subsection*{Forschungsfragen für die Dissertation}

\newlist{questions}{enumerate}{2}
\setlist[questions,1]{label=\textsf{\textbf{Q\arabic*:}},ref=RQ\arabic*}
\setlist[questions,2]{label=(\alph*),ref=\thequestionsi(\alph*)}

Die Dissertation soll die folgenden Forschungsfragen beantworten:  

\begin{questions}
    \item Wie lassen sich Botnetze während der Verbreitungs-Phase erkennen? \label{itm:q1}
    \item Wie lassen sich Botnetze während der Scan-Phase erkennen? \label{itm:q2}
    \item Welche Kombinationen von Methoden zur Botnetzerkennung führen zu einer Optimierung der Erkennung? \label{itm:q3}
    \item Welche Methoden bringen zu den jeweiligen Phasen passenden Ergebnisse? \label{itm:q4}
\end{questions}

In Frage \ref{itm:q1} und \ref{itm:q2} soll es um die Erkennung von Botnetzen während der frühen Phasen gehen, während Botnetze sich in einem Netzwerk verbreiten. Mit Frage
\ref{itm:q3} soll geklärt werden, welche Erkennungsmethoden am besten kombiniert werden können. Die Frage \ref{itm:q4} soll aus den vorherigen Ergebnissen klären, welche
Kombination von Methoden, zu welchen Phasen eingesetzt werden können. \\ \\ In Kapitel \ref{sec:theory} werden die theoretischen Hintergründe zum Thema der
Dissertation und verschiedene Prozesse zur Erkennung von Botnetzen, sowie mehrere Botnetze, die in Fallstudien eingesetzt werden sollen beschrieben. Kapitel \ref{sec:research}
erläutert den aktuellen Stand der Forschung zur Erkennung von Botnetzen, sowie zu den Themen, die in der Dissertation behandelt werden sollen. In Kapitel \ref{sec:goals} werden
die Ziele der Dissertation aufgestellt. Das Kapitel \ref{sec:methods} erklärt, wie in der Dissertation vorgegangen werden soll, um ein Konzept mit einer dazugehörigen Implementierung
zu erstellen sowie den geplanten Aufbau eines Laborexperiments und mögliche Fallstudien. Das letzte Kapitel \ref{sec:plan} stellt einen Zeitplan vor sowie die Struktur, wie 
die Dissertation aufgebaut sein soll.
