\section{Ziel der Dissertation}
\label{sec:goals}

Mit der Dissertation soll die Botnetzerkennung optimiert werden. Ziel ist es, ein Konzept zu erstellen, mit einer darauffolgenden Implementierung. Das erste Ziel soll
die Ausarbeitung und Zusammenfassung verschiedener aktueller Methoden sein, sowie deren Vor- und Nachteile. So soll sich herausarbeiten, welche Methoden miteinander
kombiniert werden können. Das darauf folgende Ziel ist die Erstellung eines Konzepts zur Botnetzerkennung, welches zu den früheren Phasen vor der Ausbreitung des 
Botnetzes, mehrere Erkennungslevel integriert, die aus einer Kombination mehrerer Erkennungsmethoden besteht. Die verschiedenen Methoden sollen zu jeder Phase separat
ausgearbeitet werden. Aus dem aktuellen Stand der Forschung zu den einzelnen Erkennungsmethoden geht hervor, 
dass die aktuellen Methoden sich eher auf die späteren Phasen eines Botnetzes fokussieren, bei denen sich das Botnetz bereits in einem Netzwerk ausgebreitet hat.
Daher ist ein weiteres Ziel, der Fokus auf die Entwicklung zur Erkennung der früheren Phasen. Ein weiteres Ziel ist, herauszufinden, ob die Erkennung auch unbekannte Botnetze 
identifizieren kann, indem Fallstudien durchgeführt werden. \\ Gesamtziel der Dissertation ist die Implementierung einer Technik zur Verbesserung der Botnetzerkennung woraus
ein Datensatz abgeleitet werden soll. Ziel der Dissertation ist es nicht, aus der Erkennung Maßnahmen zur Verteidigung gegen Botnetze zu treffen. Die Dissertation soll sich 
ausschließlich auf die Erkennung konzentrieren.
