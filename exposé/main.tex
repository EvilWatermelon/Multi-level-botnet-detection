\documentclass[
    a4paper,
    pagesize,
	pdftex,
    12pt,
]{scrartcl}

\usepackage[T1]{fontenc}
\usepackage[ngerman]{babel}

\usepackage[unicode=true]{hyperref}
\usepackage{graphicx}
\usepackage[nopatch=footnote,draft=false,babel,tracking=true,kerning=true,spacing=true]{microtype}

\usepackage{enumerate}
\usepackage{subcaption}
\usepackage{abstract}
\usepackage{amsmath}
\usepackage{mathtools}
\usepackage{listings}
\usepackage{enumitem}
\usepackage{longtable}
\usepackage[table]{xcolor}  

\graphicspath{{./pictures/}}

\begin{document}

\begin{titlepage}
    \begin{center}
        
        \includegraphics[height=25mm]{S04_HTW_Berlin_Logo_pos_FARBIG_RGB.jpg} \\
        \vspace{1.0cm}
         Mehrschichtige Erkennung von IoT-Botnetzen während der Scan-, Ausbreitungs- und Angriffsphase
             
        \vspace{1.5cm}
 
        \textbf{Exposé zur Dissertation} \\    
        zur Erlangung des Grades eines\\
        Doktors

        \vspace{1.5cm}
        am Fachbereich 4 - Informatik, Kommunikation und Wirtschaft\\
        der Hochschule für Technik und Wirtschaft Berlin\\
        
        \vspace{1.5cm}

        vorgelegt von \\
        \textbf{Jan Schröder}
      
        %\includegraphics[width=0.4\textwidth]{university}
        \vspace{1.5cm}    
        Berlin, \today\\
        
             
    \end{center}
\end{titlepage}

\pagenumbering{gobble}

\tableofcontents
\newpage

\pagenumbering{arabic}

% left fixed width:
\newcolumntype{L}[1]{>{\raggedright\arraybackslash}p{#1}}

\begin{longtable}[c]{ |L{4cm}|L{10cm}| }
    \hline
    \rowcolor{lightgray} Begriff & Erklärung \\ [0.5ex]
    \hline
    \endfirsthead

    Malware & Lustige Erklärung \\
    \hline
    Internet of Things & Noch eine lustige Erklärung \\
    \hline
    Botnetz & (\textit{englisch: botnet}) Saving it for later \\
    \hline
    Software-defined Networking & \\
    \hline
    Erkennungsindikatoren & Dieser Sammelbegriff beinhaltet Variablen, Daten, Charakteristiken, Attribute etc. die die Ausgabe eines ML Modells repräsentieren. \\
    \hline
    \caption{Diese Tabelle erläutert Begriffe, welche sowohl im Exposé als auch der späteren Dissertation Verwendung finden.\label{tab:term_table}}\\
\end{longtable}
\newpage
\section{Forschungsfragen und Aufbau des Exposés}

Botnetze im Internet of Things (IoT) Gebiet haben in der modernen Welt sehr stark an Bedeutung gewonnen. Gerade im privaten 
Haushalt finden sich immer mehr IoT-Geräte die durch Botnetze wie Mirai und Hajime infiziert werden können. Daher ist es 
sehr wichtig, dass diese Botnetze durch eine Erkennung identifiziert und abgewehrt werden. Die Dissertation soll sich mit 
der Erkennung von Botnetzen beschäftigen und die Identifikation optimieren.

\subsection*{Forschungsfragen für die Dissertation}

\newlist{questions}{enumerate}{2}
\setlist[questions,1]{label=\textsf{\textbf{Q\arabic*:}},ref=RQ\arabic*}
\setlist[questions,2]{label=(\alph*),ref=\thequestionsi(\alph*)}

Die Dissertation soll die folgenden Forschungsfragen beantworten:  

\begin{questions}
    \item Wie lassen sich Botnetze während der Verbreitungsphase erkennen? \label{itm:q1}
    \item Wie lassen sich Botnetze während der Scan-Phase erkennen? \label{itm:q2}
    \item Welche Kombinationen von bereits bekannten Methoden führen zu einer Verbesserung der Erkennung von Botnetzen? \label{itm:q3}
    \item Welche Kombinationen von bereits bekannten Techniken führen zu einer Verbesserung der Erkennung von Botnetzen? \label{itm:q4}
    \item Wie lässt sich die Erkennung von Botnetzen optimieren, wenn mehrere Erkennungsstufen verwendet werden? \label{itm:q5}
\end{questions}

In Frage \ref{itm:q1} und \ref{itm:q2} soll es um die Erkennung von Botnetzen während der frühen Phasen gehen während Botnetze sich in einem Netzwerk verbreiten. Mit Frage
\ref{itm:q3} soll geklärt werden, welche gängigen Erkennungsmethoden am besten separat funktionieren und wie diese miteinander in Verbindung gebracht werden können. Dies 
steht auch mit Forschungsfrage \ref{itm:q4} in Relation, da über die Methoden mehrere Techniken zur Erkennung in Verbindung gebracht werden sollen. Die Frage \ref{itm:q5} 
soll klären, wie die Kombination von Methoden und die Kombination von Techniken zu einer möglichen Optimierung der Botnetzerkennung führen.
\\ \\ 
In den folgenden Kapitel \ref{sec:theory} werden die theoretischen Hintergründe zum Thema der Dissertation und verschiedene Prozesse zur Erkennung von Botnetzen, sowie 
mehrere Botnetze die in Fallstudien eingesetzt werden sollen beschrieben. Kapitel \ref{sec:research} erläutert den aktuellen Stand der Forschung zur Erkennung von Botnetzen,
sowie zu den Themen die in der Dissertation behandelt werden sollen. In Kapitel \ref{sec:goals} werden die Ziele der Dissertation aufgestellt. Das Kapitel \ref{sec:methods}
erklärt wie in der Dissertation vorgegangen werden soll, um ein Konzept mit einer dazugehörigen Implementierung zu erstellen sowie den geplanten Aufbau eines Laborexperiments
und mögliche Fallstudien. Das letzte Kapitel \ref{sec:plan} stellt einen Zeitplan vor sowie die Struktur, die die Dissertation haben soll.
\newpage
\section{Stand der Forschung}
\label{sec:research}

Im Zusammenhang mit der Botnetzerkennung findet ein Großteil der Forschung zu diesem Gebiet im Zusammenhang mit Maschinellem Lernen (ML) statt wie
unter anderem zum Beispiel \cite{SAlrayes2022ModelingOB,DBLP:journals/comcom/Alani22,Habtamu2022ASR} zeigen. Wazzan et al. \cite{Wazzan2021InternetOT} schlägt die Kombination 
mehrerer Technologien vor womit sich die Dissertation auch beschäftigen soll. Daher sollen neben der Erkennung über ML auch weitere Technologien mitbetrachtet werden. Wazzan et al. 
fassen anhand verschiedener Studien folgende weitere Technologien und Architekturen zusammen: Software Defined Network (SDN), Edge Computing, Blockchain, Fog Computing und Network 
Function Virtualization (NFV). Im folgenden wird der aktuelle Stand der Forschung zu jeder dieser Technologien betrachtet. \\ \\ Zha et al. \cite{DBLP:conf/cns/Zha0GMC19} beschreiben 
eine Bibliothek, welche Botnetz Erkennung anhand eines SDNs durchführt. Bei dem Konzept des Frameworks handelt es sich darum, Botnetz und C\&C-Kanal Aktivitäten 
zu identifizieren, damit die Robustheit der Erkennung erhöht wird. Zum Aufbau gehört ein SDN Controller, welcher ein ML Modell enthält, dass über Erkennung und 
Überwachung trainiert. Die Überwachung des Netzwerks, findet über Software Switches statt, welche vollen Zugang zum Netzwerkverkehr von Virtuellen Maschinen (VM) haben.
Das ML Modell wird für das komplette Netzwerk verwendet um Bot Aktivitäten zu erkennen, während für einzelne Server, dessen VMs mit einem Software Switch kommunizieren, 
lokale Netzwerküberwachung stattfindet. Über die einzelnen Software Switches, findet die C\&C Identifizierung statt. Dabei ist speziell Peer-to-Peer Verkehr, HTTP und IRC
Protokolle im Fokus. Das ML Modell ist ein Neuronales Netzwerk (NN), welches kompromitierte Hosts findet anhand von aufgebauten Verbindungen des Hosts. Die Entscheidung, ob ein 
Host zu einem Bot wird, entscheidet das NN über einen festgelegten Schwellenwert. Mit einer einzelnen positiven Identifizierung ist aber dennoch nicht eindeutig geklärt, dass der
Host auch wirklich ein Bot ist, da er z.B. nicht unbedingt mit dem C\&C-Server kommuniziert. An dieser Stelle wird der Host dennoch dauerhaft weiter beobachtet.
Der aktuelle Stand Forschung zeigt hier, dass die Erkennung nicht während der Scan-Phase durchführt wird, womit sich ein Ziel ergibt Botnetzerkennung in frühen Botnetzphase durchzuführen,
was in der Dissertation verfolgt werden soll und weiter in Kapitel \ref{sec:goals} erläutert wird. Zusätzlich zu dem beschriebenen Framework, erläutern Negera et al. \cite{DBLP:journals/sensors/NegeraSDMA22}
verschiedene Arbeiten zur Erkennung von Botnetzen in SDNs über Modelle des ML. \\ \\ Neben der Erkennung mit SDNs findet in der aktuellen Forschung auch die Erkennung von Botnetzen über Edge Computing Architekturen 
statt, wie \cite{gromov2022edge} in ihrer Arbeit zeigen. Bei dieser Erkennung verwenden Gromov et al. in einem Edge Computing System ein Convolutional Neural Network (CNN). Der Netzwerkverkehr der einzelnen IoT-Geräte
wird an ein NVIDIA Jetson Nano Entwickler Kit zur Botnetzerkennung geleitet. Dabei werden anhand von Computer Vision Bilder zur Erkennung erzeugt, welche in auffällige und unauffällige Bilder klassifiziert werden um 
die Performance zu erhöhen sowie das Netzwerk zu verkleinern. 
\\ \\ Bei der Blockchain basierten Erkennung geht es darum eine Blockchain zu implementieren und in diesem Netzwerk ein IoT Netzwerk auf Botnetze zu überwachen. Dabei werden unterschiedliche Methoden innerhalb der Blockchain 
implementiert und mit den Vorteilen der Blockchain verwendet. Sagirlar et al. \cite{DBLP:journals/corr/abs-1809-10775} nutzen diese Methode indem sie innerhalb der Blockchain eine Peer-to-Peer Botnetz Erkennung implementieren. 
Salim et al. \cite{DBLP:journals/sensors/SalimATPP22} erläutern ein Framework, welches Digital Twins und Packet Auditor in einer Blockchain zur Erkennung einsetzen. \\ \\ Neben Blockchains wird auch Fog Computing zur 
Botnetzerkennung eingesetzt. \cite{Lawal2020AnAM} stellen ein Framework vor, welches Anomalien in einem IoT Netzwerk erkennt, aufgebaut als Fog Computing Architektur, anhand von signatur- und anomaliebasierten Erkennungsmethoden. 
Die signaturbasierte Methode greift auf eine Datenbank zu, welche IP Adressen enthält die als Angriffsursprung identifiziert sind. Die anomaliebasierte Methode klassifiziert anhand eines Extreme Gradient Booster 
(XGBoost) Algorithmus \cite{DBLP:journals/corr/ChenG16} zwischen auffälligem und unauffälligem Netzwerkverkehr. Der Netzwerkverkehr fließt zu Beginn durch ein signaturbasiertes IDS um die 
IP Adresse mit der Datenbank zu vergleichen. Sollte nichts gefunden werden, fließt der Verkehr durch ein anomaliebasiertes IDS und klassifiziert ihn entsprechend. 

% - black und whitelisting fällt immer wieder auf bei der Erkennung in Zha et al. und 

\newpage
\section{Theoretischer Hintergrund}
\label{sec:theory}

Internet of Things (IoT) ist ein Gebiet, welches dem Alltag viele Vorteile bringt. Durch die Möglichkeit
Geräte aus dem Alltag mit dem Internet zu verbinden, birgt IoT aber auch Sicherheitslücken die besagte Geräte sehr Gefährlich
werden lässt. Um dem entgegen zu wirken soll sich die Dissertation im Gebiet der Malwareforschung befinden. Im speziellen soll es dabei um 
die Erkennung von Botnetzen gehen, welche Angriffe über IoT-Geräte ausführen. \\ Ein Botnetz ist der Zusammenschluss von Hosts, auch Bots oder Zombies genannt, gesteuert von einem Angreifer, 
auch Botmaster genannt in einem Overlay-Netzwerk \cite{Xing2021SurveyOB}. Die Botnetze nutzen Zero-day Schwachstellen, Peer-2-Peer Netze, Phishing Angriffe, Anonyme Netzwerke,
Blockchain Netzwerke und Stromnetze zur Verbreitung für ihre Verwendungszwecke \cite{DBLP:conf/cycon/CasenoveM14,DBLP:conf/esorics/KurtECAU20}. Auf Basis der Architektur 
des Botnetzes findet zu jeder Zeit ein Kommunikations- und Kontrollprozess mit dem Command und Control (C\&C)-Server statt. Der C\&C-Server gibt den Bots Befehle die 
diese dann durchführen \cite{SCHILLER200729} zum Beispiel, über das Internet Relay Chat (IRC)-Protokoll. \\ Botnetze durchlaufen drei Phasen wie Wazzan et al. 
\cite{Wazzan2021InternetOT} beschreiben, scannen, ausbreiten und angreifen. Während der Scan-Phase sucht ein Bot nach vulnerablen IoT-Geräten und 
infiziert das Gerät entweder durch brute force Methoden oder durch Ausnutzen einer Schwachstelle.
In der Ausbreitungs-Phase ist eine lauffähige Version des Bots installiert und auf Basis der Architektur des infizierten Geräts ausgeführt.
Um auf dem Gerät Malware zu verhindern die nicht vom Bot selbst ausgeführt wird, stoppt der Bot andere Prozesse um Ports für sich selbst zu blocken. 
Daraufhin rekrutiert das bösartige Programm weitere Bots um das Botnetz so schnell wie möglich zu erweitern. In der Angriffs-Phase führt das Botnetz Angriffe wie Distributed Denial of 
Service (DDoS), krypto mining und spam Angriffe aus. Die erläuterten Phasen arbeiten auch Studien wie 
\cite{10.1007/978-3-030-33229-7_21, Alzahrani2020,DBLP:journals/computer/VlajicZ18,NGUYEN2020128} aus. \\ Nach der Erläuterung wie Botnetze funktionieren ist nun zu klären, wie der Prozess eines 
Botnetzes erkennbar ist um IoT-Geräte entsprechend zu schützen. Nach Xing et al. \cite{Xing2021SurveyOB} kann die Botnetz Erkennung in Honeypot Analyse, Signaturen aus der Kommunikation und 
abnormales Verhalten klassifiziert werden. Wie Abbildung \ref{fig:bot_det_met} zeigt, unterteilen diese Klassifikationen Methoden zur Erkennung. 

\begin{figure}[h!]
    \centering
    \includegraphics[scale=0.314]{./pictures/botnet_detection_methods.png}
    \caption{Klassifizierte Erkennungsmethoden von Botnetzen (übernommen von \cite{Xing2021SurveyOB}).}
    \label{fig:bot_det_met}
\end{figure}

Die \textit{Honeypot Analyse} erkennt Code Beispiele durch das Honeypot trapping was eine hohe Genauigkeit von bereits bekannten Botnetzen ermöglicht. Die Honeypot Methoden können
verschlüsselten Netzwerkverkehr nur schlecht erkennen sowie unbekannte Botnetze. Bots die eigene Funktionen zur Umgehung von Honeypots besitzen, können durch fehlende Benutzereingriffe
auch nicht von der \textit{Honeypot Analyse} erfasst werden. Weit verbreitet sind die Methoden Erkennung von \textit{Kommunikationssignaturen} anhand von Signaturen und Muster. Dabei 
werden in Intrusion Detection Systems (IDS) Regeln für den Merkmalsabgleich hinterlegt um Botnetz aktivitäten zu identifizieren. Dadurch können IDS Botnetze mit bestimmten Merkmalen 
erkennen, aber unbekannte Funktionen werden dabei nicht erkannt sowie auch Botnetze die Techniken zur Verschleierung von Code nutzen. Bei den Methoden durch Erkennung vom
\textit{abnormales Verhalten} ist die Idee, Hostverhalten- oder Netzwerkverkehr Auffälligkeiten zwischen in gutartig und bösartig zu klassifizieren. Neben den erwähnten Methoden 
erläutern Singh et al. \cite{DBLP:journals/compsec/SinghSK19} Techniken, zur Erkennung von Botnetzen. Singh et al. klassifizieren die Techniken in Flow-, Anomalie-, Flussmittel-, Domain Generation 
Algorithm (DGA)-basierten \cite{DBLP:journals/jksucis/ManasrahKF22} und Bot infizierungs Erkennung. Bei der Flow-basierten Technik findet eine Klassifizierung des Netzwerkverkehrs auf 
Basis von verschiedenen Parametern statt, wobei eine Aufteilung des Verkehrs in bösartig und gutartig stattfindet. Anhand von Parametern oder Mustern versucht die Anomalie-basierte Technik 
anomalien im Netzwerkverkehr zu finden, die sich vom regulären Verkehr unterscheiden. Über die Flussmittel-basierte Technik werden im Netzwerkverkehr IP-Flüsse gefunden. Dabei wird darauf 
geachtet, wie sich die IP-Larte verändert, welche in Relation zur einer Domäne steht und einen niedrigen Time To Live (TTL)-Wert hat. Die Konzentrationen auf abgefragte Domänen führt die 
DGA-basierte Technik durch. Mit dieser Technik soll zwischen algorithmischen bösartig erzeugten und gutartigen Domänen differenziert werden. Nach dem Stand von \cite{DBLP:journals/compsec/SinghSK19} 
versuchen die aktuelleren Technikansätze, anstatt C\&C-Server zu identifizieren, infizierte Geräte zu erkennen. \\ Die Dissertation soll sich mit der Erkennung über mehrere Level beschäftigen.
Konkret bedeutet es, die Verwendung mehrerer Techniken anhand einer Kombination aus mehreren Methoden. Die Umsetzung über mehrere Erkennungslevel erklären Stevanovic und Pedersen 
\cite{DBLP:journals/ijcysa/StevanovicP16}. Dabei führen Stevanovic und Pedersen eine Analyse des Netzwerkverkehrs durch, um die Kommunikation mit dem C\&C-Server sowie den Angriffsverkehr anhand von
TCP, UDP und DNS zu erkennen. Die Erkennung basiert auf supervised Machine Learning um bestimmte Muster zu identifizieren. Das komplette System besteht aus insgesamt drei Komponenten: die Verarbeitung,
Klassifizierung und Client Analyse. In der ersten Komponente werden der Netzwerkverkehr verarbeitet durch Analyse und Extraktion anhand von statistischen Funktionen. Abbildung \ref{fig:stats_features} 
zeigt eine Liste mit den extrahierten Informationen aus TCP und UDP. 

\begin{figure}[h!]
    \centering
    \includegraphics[scale=0.5]{./pictures/statistic_features.png}
    \caption{TCP und UDP Informationen statistisch zusammengefasst (übernommen von \cite{DBLP:journals/ijcysa/StevanovicP16}).}
    \label{fig:stats_features}
\end{figure}

\begin{figure}[h!]
    \centering
    \includegraphics[scale=0.5]{./pictures/statistic_dns.png}
    \caption{DNS Informationen statistisch zusammengefasst (übernommen von \cite{DBLP:journals/ijcysa/StevanovicP16}).}
    \label{fig:stats_dns}
\end{figure}

Bei der DNS Analyse werden Fully Qualified Domain Names (FQDN) beobachtet und für jeden FQDN, statistische Eigenschaften extrahiert, welche in Abbildung \ref{fig:stats_dns} aufgelistet sind. Zur 
Klassifizierung nutzt die Botnetz Erkennung einen Random Forest Classifier um dann über eine Client entitäten Analyse einen Report zu erstellen über infizierte Geräte. Für die Dissertation sollen 
wie bei Stevanovic und Pedersen auch mehrere Level zur Erkennung von Botnetzen eingesetzt werden. Das daraus resultierende Ziel erklärt Kapitel \ref{sec:goals} ausführlicher.
Die Botnetz Erkennung nach den unterschiedlichen Methoden und Techniken mit mehrere Level soll mehr Geräte vor der illegalen Verwendung von Botnetzen schützen. 

\subsection*{Angriffsmöglichkeiten zum Testen}
Um den Prozess der Botnetz Erkennung durch Fallstudien zu testen, sollen in der Dissertation verschiedene Botnetze Verwendung finden und alle drei Phasen durchführen. Dabei sollen bereits bekannte
Botnetze implementiert und ausgeführt werden. Zusätzlich soll ein Botnetz implementiert werden um Herauszufinden, ob das System auch neue, unbekannte Botnetze erkennt. Der Fokus bei den Botnetzen 
soll auch weiterhin im IoT Bereich liegen. Kolias et al. \cite{DBLP:journals/computer/KoliasKSV17} weisen in ihrer Arbeit neben Mirai auf das Botnetz Hajime hin, welches in der Dissertation
für die Botnetzerkennung ausgeführt werden soll. Nach Kolias et al. besteht Mirai aus vier Komponenten, dem \textit{Bot}, dem \textit{C\&C-Server}, der \textit{loader} und der \textit{report Server}. 
Der \textit{Bot} und \textit{C\&C-Server} weichen nicht von der allgemeinen Funktionsweise ab, wie in \ref{sec:theory} erklärt. Der \textit{loader} übernimmt die Kommunikation mit neuen infizierten 
Geräten und verteilt direkt an sie ausführbare Dateien. Der \textit{report Server} verwaltet Informationen über alle Geräte im Botnetz über eine Datenbank und kommuniziert mit den neu infizierten Geräten. 
Im folgenden Ablauf operiert und kommuniziert Mirai. 
Zu Beginn scannt Mirai zufällige IP Adressen über TCP ob die Ports 23 oder 2323 zuhören. Über brute-force Angriffe sucht der bot IoT Geräte, die schlecht konfiguriert sind (z.B. Standard Login Daten die
nicht geändert wurden). Mit einer geöffneten Shell gibt der Bot Informationen über das Gerät an den report Server über einen anderen Port. Der botmaster prüft über den C\&C-Server neu ausgewählte Geräte 
und anhand des report Servers den aktuellen Status des Botnetzes. Anhand der Informationen über die Geräte kann der Botmaster entsprechende Geräte zum infizieren auswählen und über ein Infect-Befehl 
über den loader ausführen. Der loader führt auf den ausgewählten Gerät Instruktionen aus zum herunterladen der Malware Binärdatei. Dabei stellt die Malware sicher, dass keine anderen Malware Programme auf
dem Gerät ausgeführt werden und schließt sowohl Secure Shell (SSH), als auch Telnet Programme. Der neue Bot bekommt über eine Domäne vom C\&C-Server nun mögliche Angriffsbefehle. Den initialen Prozess 
der Suche nach offenen Ports führt auch Hajime durch. Hajime ist ein Peer-to-Peer Netzwerk, welches auf BitTorrent's Distributed Hash Table (DHT) aufbaut \cite{DBLP:conf/ndss/HerwigHHRL19,2017AnalyzingTP}. 
BitTorrent nutzt das Kademlia Protokoll \cite{DBLP:conf/iptps/MaymounkovM02} und zusätzlich zur direkten Peer-to-Peer Kommunikation nutzt Hajime zusätzlich das uTorrent Transport Protocol. Für weitere
technische Erläuterungen zu Hajime, analysieren \cite{DBLP:conf/ndss/HerwigHHRL19} die Phasen des Botnetzes. 

% - SDN, Edge Computing etc. Referenz muss hierhin und einzelne Techniken noch erklärt werden
%------------------------------------------------------------------------------------------------------------
% - Ein ML Modell implementieren was via SDN ein IoT Netzwerk überwacht
% - Laborexperiment
% - ML Modell -> Vielleicht Unsupervised Learning, passenden Algorithmus raussuchen (basierend auf den besten Erkennungsmethoden durch ML)
% - Das Paper im Auge behalten für entsprechende Begriffe
% - WAS IST DAS ZIEL? - Welche Ergebnisse will ich am Ende haben?
% - Ist ein Zusammenhang wichtig, wofür das Botnet verwendet werden soll?
% - Ich will die Erkennung von Botnetzen verbessern (Ergebnisse sollen darauf beruhen) - Wurde die Erkennung durch meine Diss verbessert?
% - Im aktuellen Stand der Forschung auf die Hauptgebiete aus der Einführung eingehen
% - Es geht hierbei nur um die Erkennung, keine Maßnahmen und keine Verteidigung (vielleicht am Ende).
\newpage
\section{Stand der Forschung}
\label{sec:research}

Im Zusammenhang mit der Botnetzerkennung findet ein Großteil der Forschung zu diesem Gebiet im Zusammenhang mit Maschinellem Lernen (ML) statt wie
unter anderem zum Beispiel \cite{SAlrayes2022ModelingOB,DBLP:journals/comcom/Alani22,Habtamu2022ASR} zeigen. Wazzan et al. \cite{Wazzan2021InternetOT} schlägt die Kombination 
mehrerer Technologien vor womit sich die Dissertation auch beschäftigen soll. Daher sollen neben der Erkennung über ML auch weitere Technologien mitbetrachtet werden. Wazzan et al. 
fassen anhand verschiedener Studien folgende weitere Technologien und Architekturen zusammen: Software Defined Network (SDN), Edge Computing, Blockchain, Fog Computing und Network 
Function Virtualization (NFV). Im folgenden wird der aktuelle Stand der Forschung zu jeder dieser Technologien betrachtet. \\ \\ Zha et al. \cite{DBLP:conf/cns/Zha0GMC19} beschreiben 
eine Bibliothek, welche Botnetz Erkennung anhand eines SDNs durchführt. Bei dem Konzept des Frameworks handelt es sich darum, Botnetz und C\&C-Kanal Aktivitäten 
zu identifizieren, damit die Robustheit der Erkennung erhöht wird. Zum Aufbau gehört ein SDN Controller, welcher ein ML Modell enthält, dass über Erkennung und 
Überwachung trainiert. Die Überwachung des Netzwerks, findet über Software Switches statt, welche vollen Zugang zum Netzwerkverkehr von Virtuellen Maschinen (VM) haben.
Das ML Modell wird für das komplette Netzwerk verwendet um Bot Aktivitäten zu erkennen, während für einzelne Server, dessen VMs mit einem Software Switch kommunizieren, 
lokale Netzwerküberwachung stattfindet. Über die einzelnen Software Switches, findet die C\&C Identifizierung statt. Dabei ist speziell Peer-to-Peer Verkehr, HTTP und IRC
Protokolle im Fokus. Das ML Modell ist ein Neuronales Netzwerk (NN), welches kompromitierte Hosts findet anhand von aufgebauten Verbindungen des Hosts. Die Entscheidung, ob ein 
Host zu einem Bot wird, entscheidet das NN über einen festgelegten Schwellenwert. Mit einer einzelnen positiven Identifizierung ist aber dennoch nicht eindeutig geklärt, dass der
Host auch wirklich ein Bot ist, da er z.B. nicht unbedingt mit dem C\&C-Server kommuniziert. An dieser Stelle wird der Host dennoch dauerhaft weiter beobachtet.
Der aktuelle Stand Forschung zeigt hier, dass die Erkennung nicht während der Scan-Phase durchführt wird, womit sich ein Ziel ergibt Botnetzerkennung in frühen Botnetzphase durchzuführen,
was in der Dissertation verfolgt werden soll und weiter in Kapitel \ref{sec:goals} erläutert wird. Zusätzlich zu dem beschriebenen Framework, erläutern Negera et al. \cite{DBLP:journals/sensors/NegeraSDMA22}
verschiedene Arbeiten zur Erkennung von Botnetzen in SDNs über Modelle des ML. \\ \\ Neben der Erkennung mit SDNs findet in der aktuellen Forschung auch die Erkennung von Botnetzen über Edge Computing Architekturen 
statt, wie \cite{gromov2022edge} in ihrer Arbeit zeigen. Bei dieser Erkennung verwenden Gromov et al. in einem Edge Computing System ein Convolutional Neural Network (CNN). Der Netzwerkverkehr der einzelnen IoT-Geräte
wird an ein NVIDIA Jetson Nano Entwickler Kit zur Botnetzerkennung geleitet. Dabei werden anhand von Computer Vision Bilder zur Erkennung erzeugt, welche in auffällige und unauffällige Bilder klassifiziert werden um 
die Performance zu erhöhen sowie das Netzwerk zu verkleinern. 
\\ \\ Bei der Blockchain basierten Erkennung geht es darum eine Blockchain zu implementieren und in diesem Netzwerk ein IoT Netzwerk auf Botnetze zu überwachen. Dabei werden unterschiedliche Methoden innerhalb der Blockchain 
implementiert und mit den Vorteilen der Blockchain verwendet. Sagirlar et al. \cite{DBLP:journals/corr/abs-1809-10775} nutzen diese Methode indem sie innerhalb der Blockchain eine Peer-to-Peer Botnetz Erkennung implementieren. 
Salim et al. \cite{DBLP:journals/sensors/SalimATPP22} erläutern ein Framework, welches Digital Twins und Packet Auditor in einer Blockchain zur Erkennung einsetzen. \\ \\ Neben Blockchains wird auch Fog Computing zur 
Botnetzerkennung eingesetzt. \cite{Lawal2020AnAM} stellen ein Framework vor, welches Anomalien in einem IoT Netzwerk erkennt, aufgebaut als Fog Computing Architektur, anhand von signatur- und anomaliebasierten Erkennungsmethoden. 
Die signaturbasierte Methode greift auf eine Datenbank zu, welche IP Adressen enthält die als Angriffsursprung identifiziert sind. Die anomaliebasierte Methode klassifiziert anhand eines Extreme Gradient Booster 
(XGBoost) Algorithmus \cite{DBLP:journals/corr/ChenG16} zwischen auffälligem und unauffälligem Netzwerkverkehr. Der Netzwerkverkehr fließt zu Beginn durch ein signaturbasiertes IDS um die 
IP Adresse mit der Datenbank zu vergleichen. Sollte nichts gefunden werden, fließt der Verkehr durch ein anomaliebasiertes IDS und klassifiziert ihn entsprechend. 

% - black und whitelisting fällt immer wieder auf bei der Erkennung in Zha et al. und 

\newpage
\subsection{Research Questions}

\newlist{questions}{enumerate}{2}
\setlist[questions,1]{label=\textsf{\textbf{Q\arabic*:}},ref=RQ\arabic*}
\setlist[questions,2]{label=(\alph*),ref=\thequestionsi(\alph*)}

\begin{questions}
    \item Welche hybride Erkennungsmethoden erreichen die beste Genauigkeit für jede einzelnen Phase?
    \item Wie sinnvoll ist eine Relation zwischen den hybriden Erkennungsmethoden jeder einzelnen Phase?
    \item Welche IoT-Netzwerkarchitektur ist am besten für eine hybride IoT-Botnetzerkennung für jede Botnetzphase geeignet?
    \item Wie lässt sich die hybride IoT-Botnetzerkennung für jede Phase ohne zusätzliche Hardware implementieren?
\end{questions}

\subsection{Hypothesen}

\newlist{hypotheses}{enumerate}{2}
\setlist[hypotheses,1]{label=\textsf{\textbf{H\arabic*:}},ref=H\arabic*}
\setlist[hypotheses,2]{label=(\alph*),ref=\thehypothesesi(\alph*)}

\begin{hypotheses}
    \item
\end{hypotheses}
\newpage
\section{Forschungsdesign und Methodik}
\label{sec:methods}

Zu Beginn soll eine Ausarbeitung und Zusammenfassung von Methoden zur Botnetzerkennung durchgeführt werden. Anhand der Zusammenfassung soll 
erkennbar werden, welche Methoden die bestmöglichen Ergebnisse liefern und welche Kombinationen von Methoden am plausibelsten sind.
Daraus soll ein Konzept gebildet werden, welches Hybride Erkennungsmethoden für die frühen Phasen eines Botnetzes in der Theorie beschreibt.
Anschließend soll mit einem Laborexperiment das Konzept implementiert werden. Das Laborexperiment soll ein lokales Netzwerk mit gängigen Smart Home 
Geräten darstellen, die in einem privaten Haushalt verwendet werden. Um die Erkennung eines Botnetzes zu testen sollen verschiedene typische 
IoT-Geräte aufgebaut werden, die von Botnetzen wie zum Beispiel Mirai angegriffen werden. Dazu ist es nötig, dass entsprechende Geräte ausgewählt werden, 
die von bekannten Botnetzen infiziert werden können. \\ \\ In einer Fallstudie ist dann vorgesehen, Botnetze auszuführen, um entsprechende Daten
zu sammeln, damit eine aussagekräftige Auswertung zu plausiblen Ergebnissen führt. Um die Ergebnisse aus der Fallstudie überprüfen
zu können, sollen diese mit Ergebnissen aus anderen vergleichbaren Arbeiten in Relation gesetzt werden. \\ Wie schon in \ref{sec:theory}
erläutert, sollen unter anderem Mirai und Hajime eingesetzt werden. Im optimalen Fall, soll für die Dissertation auch ein eigenes Botnetz
implementiert werden, um zu prüfen, ob die Botnetz Erkennung auch unbekannte Botnetze erkennt.

\newpage
\section{Zeitplan und Struktur der Dissertation}
\label{sec:plan}

Die voraussichtliche Dauer der Dissertation beträgt 4 Jahre und 6 Monate.

\begin{center}
    \begin{tabular}[c]{| c | c |}
        \hline
        \rowcolor{lightgray} Zeit & Vorgehen \\ [0.5ex]
        \hline
        März 2023 & Grundlagen, verwandte Arbeiten erläutern \\
        \hline
        August 2023 &  Laborexperiment starten, Konzept in der Dissertation beschreiben \\
        \hline
        Januar 2025 & Implementierung der IoT Geräte beschreiben \\
        \hline
        März 2025 & Fallstudien beschreiben \\
        \hline
        Juli 2025 & Ergebnisse erläutern, Fazit schreiben \\
        \hline
        Februar 2026 & Einleitung schreiben, Kontrolle der Dissertation \\
        \hline
        Juli 2026 & Abgabe \\
        \hline
    \end{tabular}
\end{center}

Zum Ende der Arbeitszeit soll die Dissertation über eine externe Stelle auf Plagiate, Rechtschreibung etc.
geprüft werden.

\subsection*{Kapitelstruktur der Dissertation}

\begin{enumerate}[label=\Roman*]
    \item Einführung in die Dissertation
    \item Grundlagen und Verwandte Arbeiten
    \item Konzept der Mehrstufigen Botnet Erkennung
    \item Implementierung des Experiments und der Fallstudien
    \item Evaluierung und Ergebnisse
\end{enumerate}
\newpage

\bibliographystyle{ieeetr}
\bibliography{references}

\end{document}
